%\documentclass{sig-alternate-10pt}
\documentclass[10pt]{sigalternate052015}

\usepackage{amsmath}
\usepackage{amssymb}
\usepackage{xcolor}
\usepackage{url}
\usepackage{times}
\usepackage{xspace}
\usepackage{listings}
\usepackage{code}
\usepackage{algorithm}
\usepackage[noend]{algpseudocode}
\usepackage{tikz}
\usetikzlibrary{arrows,automata,positioning}
\let\proof\relax
\let\endproof\relax
\usepackage{amsthm}
\usepackage{subcaption}
%\usepackage{geometry}

\usepackage{balance}


\definecolor{princetonorange}{RGB}{255,143,0}
\definecolor{tmlblue}{RGB}{0,58,120}  % tmlblue == Toronto Maple Leafs Blue!

\newcommand{\todo}[1]{\textcolor{red}{[TODO: #1]}}
\newcommand{\ratul}[1]{\textcolor{blue}{[ratul: #1]}}
\newcommand{\ryan}[1]{\textcolor{green}{[ryan: #1]}}
\newcommand{\dpw}[1]{\textcolor{tmlblue}{[dpw: #1]}}
\newcommand{\todd}[1]{\textcolor{princetonorange}{[todd: #1]}}

\newcommand{\EG}{\emph{e.g.}}
\newcommand{\IE}{\emph{i.e.}}
\newcommand{\ETC}{\emph{etc.}}
\newcommand{\ETAL}{\emph{et al.}}

\newcommand{\sysname}{{\small \sf Propane}\xspace}

\newcommand{\para}[1]{\paragraph*{\textbf{#1}}}

\newcommand{\set}[1]{\ensuremath{\{ #1 \} }}
\newcommand{\abs}[1]{\ensuremath{ \lvert #1 \rvert }}

\newcommand{\CD}[1]{\texttt{\small #1}}  % code font
\newcommand{\KW}[1]{\texttt{\small\bfseries{#1}}}
%\newcommand{\KW}[1]{\texttt{#1}}
\newcommand{\True}{\CD{true}}
\newcommand{\Define}{\KW{define}}
\newcommand{\Prefer}{\texttt{>>}}
\newcommand{\Path}{\texttt{=>}}
\newcommand{\Link}{\texttt{->}}
\newcommand{\Agg}{\KW{agg}}
\newcommand{\Any}{\KW{any}}
\newcommand{\None}{\KW{drop}}
\newcommand{\In}{\KW{in}}
\newcommand{\Out}{\KW{out}}
\newcommand{\AND}{\texttt{\&}}
\newcommand{\OR}{\texttt{|}}
\newcommand{\NOT}{\texttt{!}}
\newcommand{\Intersect}{\ensuremath{\cap}}
\newcommand{\Union}{\ensuremath{\cup}}

\newcommand{\Exit}{\KW{exit}}
\newcommand{\End}{\KW{end}}
\newcommand{\Start}{\KW{start}}
\newcommand{\Enter}{\KW{enter}}
\newcommand{\Later}{\KW{later}}
\newcommand{\Before}{\KW{before}}
\newcommand{\Internal}{\KW{internal}}
\newcommand{\Never}{\KW{never}}
\newcommand{\Only}{\KW{only}}
\newcommand{\Through}{\KW{through}}
\newcommand{\LinkKW}{\KW{link}}
\newcommand{\PathKW}{\KW{path}}
\newcommand{\Novalley}{\KW{novalley}}

%\let\Url\url
%\renewcommand*{\LaTeX}{\oldLaTeX\space}
\renewcommand{\path}[2]{ #1 \mapsto \ensuremath{#2} }



\begin{document}

\CopyrightYear{2016} 
\setcopyright{acmcopyright}
\conferenceinfo{SIGCOMM '16,}{August 22-26, 2016, Florianopolis , Brazil}
\isbn{978-1-4503-4193-6/16/08}\acmPrice{\$15.00}
\doi{http://dx.doi.org/10.1145/2934872.2934909}



%\special{papersize=8.5in,11in}
%\setlength{\pdfpageheight}{\paperheight}
%\setlength{\pdfpagewidth}{\paperwidth}

\title{Don't Mind the Gap:  Bridging Network-wide Objectives and Device-level Configurations}

%\author{Paper \#324, 14 pages}


\author{%
Ryan Beckett\\
  \affaddr{Princeton}
  %\\
  %\email{rbeckett@princeton.edu}
\and
Ratul Mahajan\\
  \affaddr{Microsoft}
  %\\
  %\email{ratul@microsoft.com}
\and
Todd Millstein\\
  \affaddr{UCLA}
  %\affaddr{University of California, Los Angeles}
  %\\
  %\email{todd@cs.ucla.edu}
\and
Jitendra Padhye\\
  \affaddr{Microsoft}
  %\\
  %\email{padhye@microsoft.com}
\and
David Walker\\
  \affaddr{Princeton}
  %\\
  %\email{dpw@princeton.edu}
}

\maketitle



\textbf{Abstract---}
We develop \sysname, a language and compiler to help network operators with a challenging, error-prone task---bridging the gap between network-wide routing objectives and low-level configurations of devices that run complex, distributed protocols.
%
The language allows operators to specify their objectives naturally, using
high-level constraints on both the shape and relative preference of traffic paths.
%
The compiler automatically translates these specifications to router-level BGP configurations, using an effective intermediate representation that compactly encodes the flow of routing information along policy-compliant paths.
%It uses We introduce data structures and algorithms that
It guarantees that the compiled configurations correctly implement the specified policy under all possible combinations of failures.
%
We show that \sysname can effectively express the policies of datacenter and backbone networks of a large cloud provider;  and despite its strong guarantees, our compiler scales to networks with hundreds or thousands of routers.



%=====================================================
%
%
%  **Keywords**
%
%
%=====================================================


\vspace{0.1in}
\noindent
\textbf{CCS Concepts}\\
$\bullet$ Networks $\rightarrow$ {\em Network control algorithms; Network reliability; Network management;} 
$\bullet$ Software and its engineering $\rightarrow$ {\em Automated static analysis; Domain specific languages}

\vspace{0.1in}
\noindent
\textbf{Keywords}\\
Propane; Domain-specific Language; BGP; Synthesis; Compilation; Fault Tolerance; Distributed Systems


\section{Outline}

\begin{itemize}
\item highlight the differences between openflow and
  \item interdomain setting  difference .  same abstraction both
    inside and outside.
\end{itemize}

\section{A Quick History}

When Software-Defined Networking (SDN) burst onto the scene in 2008,
it offered a new means to control networking hardware from a single
centralized vantagepoint.  Neither industrial engineers nor academic
researchers would be bound to the slowly-evolving technology provided by
traditional routers.  Instead, for many in the networking
community, SDN technology opened up the opportunity to program new
routing algorithms that use network-wide data to optimize performance.
The decisions made by these new algorithms could then be realized by
pushing low-level packet-processing rules out to a distributed
collection of OpenFlow-enabled switches.

While many networking researchers examined \emph{what} new algorithms
could be implementing using SDN technology, a few, in collaboration
with some programming languages folks, began to explore \emph{how}
best to express these algorithms.  With a plethora of new algorithms
and much new infrastructure being developed, there are also bound to
be many bugs, which in the short term at least, might make such network
services less reliable.  Hence, it made sense to consider
whether there might be new programming paradigms that, through their
design, might be able to cut down on certain classes of software
errors, just as Java, or other safe, garbage-collected languages cut
down on memory errors relative to C.

The first foray into this new space involved development of the
Frenetic language and system~\cite{frenetic}.  Frenetic responded to control
events from switches, such as failures or traffic statistics, and used
the information it received to compute declarative dataplane
policies.  These policies specified much of the same information as
lists of OpenFlow rules, but unlike OpenFlow, they were designed to be composable.  For
instance, one could specify a network monitoring policy separately
from a network routing policy and then implement both policies
simultaneously, without fear of conflicts, by invoking a single
function.  Under the covers, the Frenetic compiler
generated sets of OpenFlow rules and installed them
``consistently''~\cite{consistent-updates} to avoid transient bad
behaviors.  Support for composition and for consistent update
effectively raised the level of abstraction at which
programmers could implement policy.

Inspired by Frenetic, researchers worked to extend the set of
high-level abstractions available to network engineers.  For instance,
NetCore~\cite{netcore} generalized Frenetic's
simple packet patterns, allowing users to classify traffic
using more general boolean predicates over packets, and demonstrated
it was possible to compile
such programs to OpenFlow.
Pyretic~\cite{pyretic} extended NetCore's policy language to allow a
new kind of \emph{sequential composition} to complement Frenetic's
\emph{parallel composition}).  FatTire~\cite{fattire} went a step
further, adding the
abstraction of \emph{paths}, specified by regular expressions.
These paths represented the first real \emph{network-wide}
abstraction.
%FatTire also introduced the idea of specifying the
%number of back-paths required for fault tolerance and implemented such
%requirements using OpenFlow 1.3's fast-failover mechanisms.
% A year later, NetKAT~\cite{netkat} laid out a more general design still in which
% regular expressions for describing paths, could be interleaved with
% boolean predicates for classifying the traffic that should follow
% those paths.  Indeed, by connecting these various network language
% designs with the well-established theory of Kleene Algebra with
% Tests~\cite{kozen:kat}, network programming languages acquired
% rich equational and semantic theories for reasoning about the
% correctness of data plane programs.

To summarize, in the five-year period between 2008 and 2013,
researchers had come to agree upon two essential elements of stateless,
network-wide data
plane programming:  (1) regular expressions to describe paths and (2)
boolean predicates to classify packets and to choose the paths along
which to send them.  In 2014, Anderson \emph{et al.}~\cite{netkat}
recognized that network programming languages possessing this
combination of features were instances of a well-known algebraic
structure called a Kleene Algebra with Tests
(KAT)~\cite{kat}.  By identifying this connection, a rich semantic
theory and a host of algorithmic techniques could be adapted from the
world of formal algebra and applied to the study of network programs.

\section{Propane's Contributions}

While SDN was the hot topic of the day in both academia and industry,
most networks, both big and small, continued to be implemented using
traditional routers running conventional distributed protocols such as
OSPF and BGP.  Even modern data centers, which might use SDN around
the edge, often continued to use conventional routers through their
core.  In part, continued use of traditional routers is certainly
simply a legacy software and hardware issue.  It is also likely that
traditional routers, which have the benefit of being plug-and-play,
simply work well enough from a performance perspective in many common
cases.  Finally, regardless of whether a network uses SDN internally
or not, it is necessary to communicate with peers, and the only
current means of doing so is to use BGP.  Hence, interfacing with
traditional distributed protocols remains essential, even in the SDN world.

Unfortunately, management of traditional networks continues to be a
challenging task.  Indeed, one regularly hears of major networks, some
of them engineered by the most competent tech firms in the business,
suffering outages due to changes in network misconfiguration.  We
speculate that one reason for that may be the gap between the
intentions of network engineers and the specifications mechanisms
available to network operators.  On the one hand, objectives are
phrased in terms of network-wide properties, such as reachability or
access to servers; on the other hand the mechanisms use to implement
those properties are device-local.

Familiar with the history of the development of SDN abstractions, we
asked ourselves whether such abstractions could be used to program
conventional routers instead of Hence, the key question we asked 

So the essential question we asked when beginning the
development of Propane was whether the high-level abstractions
developed for specifying SDN data plane routes could be used instead to
specify control plane mechanics and whether it would be possible to
compile such specifications to legacy devices so commodity hardware
and software could be reused.

\section{Continuing Challenges}

While work 

\begin{itemize}
\item scaling and incrementality.
\item multiple protocols
\item bridging the gap
\end{itemize}

\section{A Key Lesson:  The Importance of Small Models}
\label{sec:conclusions}

The programming languages that software engineers use on a day-to-day
basis are extremely complicated.  This is true of C++, Python, Haskell, and
JavaScript and it is also true of Cisco IOS.  In order to understand
how these languages work in a deep sense, how to program with them,
and how to build reliable abstractions on top of them, one cannot
tackle the whole language all at once.  It is necessary to build
small, simple models first---little idealized languages---which might
not have every bell and whistle, but that are more uniform than their
real-world counterparts.  Such
small models will almost always leave out essential elements of the
real language, which will invariably frustrate current practioners.
However, that is
not a bug though, it is a feature---it frees the researcher to focus on
what is \emph{left in} the small model and to ignore the complexity left out.

In a sense, OpenFlow 1.0 was a small model for the world of
switchware.  It provided an understanding of the network dataplane so
simple that even programming language researchers could understand it!
Once they did, they could begin to manipulate, transform, extend and
finally apply the lessons learned to more expansive settings.  The
latter step is where Propane comes in as it adopted the ideas to the
more complex world of the traditional router.  Going forward, we hope
to see the networking community develop more small models other
components of the networking stack.

\para{Acknowledgments}
We thank R. Aditya, George Chen, and Lihua Yuan for feedback on the work and the SIGCOMM reviewers for comments on the paper. This work is supported in part by the National Science Foundation awards CNS-1161595 and CNS-1111520 as well as a gift from Cisco.


\balance

\bibliographystyle{abbrv}
\bibliography{references}

% The bibliography should be embedded for final submission.


\end{document}
